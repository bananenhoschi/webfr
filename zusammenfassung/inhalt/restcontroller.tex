\section{REST}
\subsection{RESTFullness}
\begin{itemize}
	\item Stufe 0 - RESTless -> kein REST Support
	\item Stufe 1 - Ressourcen -> Jede identifizierbare Ressource hat eine eigene URI.
	\item Stufe 2 - HTTP Verben -> GET, POST, HEAD, PUT, DELETE und OPTIONS werden gemäss ihrer
Spezifikation unterstützt.
	\item Stufe 3 – Hypermedia -> Verknüpfung von Ressourcen und Repräsentationen durch Hyperlinks.
\end{itemize}

\subsection{Cross-Origin Resource Sharing (CORS)}
Cross-Origin Resource Sharing (CORS)
CORS ist ein Mechanismus, um Webbrowsern oder auch anderen Webclients Cross-Origin-Requests zu ermöglichen. Zugriffe dieser Art sind normalerweise durch die Same-Origin-Policy untersagt. CORS ist ein Kompromiss zugunsten grösserer Flexibilität im Internet unter Berücksichtigung möglichst hoher Sicherheitsmassnahmen.
CORS definiert eine Reihe von neuen Headern im HTTP- Protokoll, die es dem Server erlauben gezielt REST-Aufrufe für bestimmte Domains zuzulassen.

\subsubsection{Preflight Request}

Durch den Preflight Request werden die CORS (Cross-Origin Resource Sharing)-Regeln für den Serverdienst abgefragt, bevor die tatsächliche Anforderung gesendet wird.
\begin{itemize}
	\item Ein Webbrowser oder ein anderer Benutzer-Agent sendet eine Preflight- Anforderung, die die Ursprungsdomäne, die Methode und die Header der tatsächlichen Anforderung umfasst.
	\item Wenn CORS für den Serverdienst aktiviert wird, dann wertet der Serverdienst die Preflight-Anforderung anhand der CORS-Regeln aus und akzeptiert die Anforderung bzw. weist sie zurück.
\end{itemize}

\subsection{Async Kommunikation}
\subsubsection{Synchrone Kommunikation}
Der Client wartet nach Absenden der Anfrage an den Server so lange, bis er eine Rückantwort erhält, und kann dann erst andere Aktivitäten ausführen.
\subsubsection{Asynchrone Kommunikation} 
Der Client sendet die Anfrage an den Server und arbeitet sofort weiter. Beim Eintreffen der Rückantwort werden dann bestimmte Routinen beim Client aufgerufen.
Die durch den Server initiierten Rückrufe (Callbacks) erfordern zusätzliche Kontrolle beim Client, wenn ungewünschte Unterbrechungen der Arbeit vermieden werden sollen.

\textbf{Vorteile}
\begin{itemize}
	\item Sendender Prozess kann weiterarbeiten, noch während die Nachricht übertragen wird.
	\item Es kann ein höherer Grad an Parallelität erreicht werden.
	\item Die Entkoppelung von Sender und Empfänger ist stärker, d.h. grösser Unabhängigkeit beider Komponenten.	
\end{itemize}

\textbf{Nachteile}
\begin{itemize}
	\item Der Sender weiss nicht, ob / wann die Nachricht angekommen ist. Das Debugging der Anwendung wird anspruchsvoller.
	\item Betriebssystem muss Puffer verwalten.
\end{itemize}

\subsection{Event Loop}

JavaScript basiert auf einem Event Loop
\begin{itemize}
	\item $"$queue.waitForMessage()$"$ wartet synchron auf eine Message
	\item Sobald Message vorhanden, wird die entsprechende Callback-Funktion auf den Stack gelegt und aufgerufen.
	\item Die Callback-Funktion wird komplett abgearbeitet.
	\item Erst wenn Callback-Funktion beendet ist, kann der der Event-Loop die Queue auf die nächste Message testen.
\end{itemize}

EventLoop ist Single-Threaded

\textbf{Vorteile}
\begin{itemize}
	\item Einfach, vor allem für die GUI-Programmierung
	\item Event und Rendering werden sequentiell ausgeführt
	\item Keine synchronisierung der Threads notwendig
	\item Typisches Programmiermodell für GUI-Entwicklung (z.B. auch in Android)
\end{itemize}
\textbf{Nachteile:}
\begin{itemize}
	\item Aufwändige, zeitintensive Aufgaben können GUI blockieren, dadurch schlechtes User Erlebnis
\end{itemize}

Darum sollten zeitintensive Aufgaben asynchron programmiert werden (Ajax: Asynchronous JavaScript and XML)

\subsection{Rest Controller}

\begin{minted}[breaklines]{java}
@CrossOrigin
@RestController
@RequestMapping("/questionnaires")
@RequiredArgsConstructor
@Slf4j
public class QuestionnaireController {

    private final QuestionnaireRepository questionnaireRepository;
    
    @GetMapping
    public ResponseEntity<List<Questionnaire>> findAll() {
        Sort sort = Sort.by(Sort.Direction.ASC, "title");
        return new ResponseEntity<>(questionnaireRepository.findAll(sort), HttpStatus.OK);
    }

    @GetMapping("/{id}")
    public ResponseEntity findById(@PathVariable String id) {
        log.info("Find by id {}", id);
        Optional<Questionnaire> questionnaire = questionnaireRepository.findById(id);
        if (questionnaire.isPresent()) {
            log.info("Find a Questionnaire for id {}", id);
            return new ResponseEntity(questionnaire.get(), HttpStatus.OK);
        }
        log.error("No Questionnaire found for id {}", id);
        return new ResponseEntity<>(HttpStatus.NOT_FOUND);
    }

    @PostMapping
    public ResponseEntity<Questionnaire> create(@Valid @RequestBody Questionnaire questionnaire, BindingResult result) {
        if (result.hasErrors()) {
            return new ResponseEntity<>(HttpStatus.PRECONDITION_FAILED);
        }
        questionnaire = questionnaireRepository.save(questionnaire);
        log.debug("Created questionnaire with id=" + questionnaire.getId());
        return new ResponseEntity<>(questionnaire, HttpStatus.CREATED);
    }

    @PutMapping(value = "/{id}")
    public ResponseEntity<Questionnaire> update(@PathVariable String id,
                                                @Valid @RequestBody Questionnaire questionnaire, BindingResult result) {
        if (result.hasErrors()) {
            return new ResponseEntity<>(HttpStatus.PRECONDITION_FAILED);
        }
        Optional<Questionnaire> questionnaireOptional = questionnaireRepository.findById(id);
        if (questionnaireOptional.isPresent()) {
            Questionnaire updateQuestionnaire = questionnaireOptional.get();
            if (questionnaire.getTitle() != null) {
                updateQuestionnaire.setTitle(questionnaire.getTitle());
            }
            if (questionnaire.getDescription() != null) {
                updateQuestionnaire.setDescription(questionnaire.getDescription());
            }
            questionnaire = questionnaireRepository.save(updateQuestionnaire);
            log.debug("Updated questionnaire with id=" + updateQuestionnaire.getId());
            return new ResponseEntity<Questionnaire>(updateQuestionnaire, HttpStatus.OK);
        }
        log.debug("No questionnaire with id=" + id + " found");
        return new ResponseEntity<>(HttpStatus.NOT_FOUND);
    }

    @DeleteMapping(value = "/{id}")
    public ResponseEntity<String> delete(@PathVariable String id) {
        Optional<Questionnaire> questionnaireOptional = questionnaireRepository.findById(id);
        if (questionnaireOptional.isPresent()) {
            questionnaireRepository.deleteById(id);
            log.debug("Deleted questionnaire with id=" + id);
            return new ResponseEntity<>(HttpStatus.OK);
        }
        log.debug("No questionnaire with id=" + id + " found");
        return new ResponseEntity<>(HttpStatus.NOT_FOUND);
    }
}	
\end{minted}

