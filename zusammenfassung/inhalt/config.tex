\section{Konfiguration}

Konfiguration über die Environment ".env": Die Environment wird in spezifischen Files festgelegt, wie ".env", ".env.development", ".env.production". Diese Files werden in unterschiedlichen Entwicklungsphasen von Tools wie webpack (automatisch vom Tool "Create React App" integriert) gelesen – aber immer bei einem Build der Applikation. Das bedeutet, dass zur Laufzeit keine Aktualisierung der Konfigurationseinstellungen möglich ist. Beachten muss man auch die Vorgaben von React für die Namensgebung der Variablen (siehe https://create-react-app.dev/docs/adding-custom-environment-variables)

Konfiguration über ein JavaScript-File: Um die Konfiguration zur Laufzeit zu ändern, muss die App beim Laden entsprechende Informationen und Anweisungen ausführen. Bei einer SPA-Applikation wird das File "index.html" geladen und verarbeitet. Ein Eintrag <script src="config.js"/> führt z.B. zum Ausführen des Skripts "config.js" und man kann z.B. eine entsprechende globalen Variable setzen. Alle globalen Objekte werden in eine Browserapplikation automatisch an die "window"-Instanz angehängt. Diese Variante hat jedoch in Bezug auf die Sicherheit seine berechtigten Nachteile, da einfach ausführbarer Code in die Applikationen integriert werden kann!

Konfiguration über ein JSON-File: Die Konfigurationseinstellungen werden am einfachsten in JSON formuliert. Dieses Format kann mit JavaScript einfach verarbeitet werden. Das Konfigurationsfile, z.B. "application.json" wird auf dem Server liegen, wo die Einstellungen vorgenommen werden können. Beim Starten der Client Applikation im Browser muss dieses Konfigurationsfile von der App geladen werden, um die entsprechenden Einstellungen zu setzen. Das bedeutet, dass bei der Initialisierung der App entsprechende Logik ausprogrammiert werden muss.