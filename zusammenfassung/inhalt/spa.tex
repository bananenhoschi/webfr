\section{Single Page Application}
\subsection{Klassische Webapplikation}

\textbf{Vorteile:}\\

\begin{itemize}
	\item Full Control
	\item Sicherheit
\end{itemize}

\textbf{Nachteile:}\\

\begin{itemize}
	\item User Experience
	\item Server-Roundtrip
\end{itemize}

\subsection{Klassische Webapplikation}

\textbf{Vorteile:}\\

\begin{itemize}
	\item User Feedback
	\item Client- vs. Server-Logik
\end{itemize}

\textbf{Nachteile:}\\

\begin{itemize}
	\item Browser-Unterstützung
	\item Verschiedene Technologien
\end{itemize}

\subsection{Definition Single Page Application (SPA)}
Eine Single Page Application ist eine Webanwendung, die keinen Seitenwechsel (Roundtrip) durchführt, sondern die Anzeige nur durch Austausch von Seitenelementen via Javascript/DOM verändert. Es gibt dabei also keine serverseitige Seitennavigation. Die URL ändert sich grundsätzlich nicht (kann aber simuliert werden!).
Initial wird eine komplette Seite oder zumindest das Grundgerüst einer Webseite vom Server geladen. Die Seite lädt anschliessend Daten über Webservices (meist REST-basierte Dienste) nach und erzeugt die Darstellung clientseitig (clientseitiges Rendern).
Eine Single Page Application wirkt damit wie eine Desktopanwendung.
