\section{Final Flashcard}

\subsection{index.js}

\inputminted{javascript}{src/index.js}

\subsection{App.js}

\begin{minted}{javascript}
import React, { useEffect } from 'react'
import { useSelector, useDispatch } from 'react-redux'
import _ from 'lodash'
import { Container } from 'reactstrap'

import Header from './Header'
import QuestionnaireContainer from '../questionnaire/QuestionnaireContainer'
import Message from './Message'
import Footer from './Footer'
import doFetch from '../network/NetworkUtil'

/**
 * Die Wurzel der React-App.
 */
const App = () => {

    const config = useSelector(state => state.config, _.isEqual)
    const error = useSelector(state => state.error, _.isEqual)
    const message = useSelector(state => state.message, _.isEqual)
    const dispatch = useDispatch()

    const readConfig = () => {
        dispatch(
            doFetch({
                url: 'application.json',
                actionType: 'CONFIG'
            })
        )
    }

    useEffect(readConfig, [])

    const renderQuestionnaireContainer = config => 
        config ? <QuestionnaireContainer serverUrl={ `${ config.url }/questionnaires` } /> : null

    const renderMessage = () =>
        error ? <Message message={ message } /> : null 

    return <Container>
        <Header title='Flashcard Client with React' subtitle='Version 3' />
        { renderQuestionnaireContainer(config) }
        { renderMessage() }
        <Footer message='@ The FHNW Team' />
    </Container>
}

export default App
\end{minted}

\subsection{Header.js}
\begin{minted}{javascript}
import React from 'react'
import { Jumbotron } from 'reactstrap'

const Header = ({ title, subtitle }) => 
    <Jumbotron >
        <h1>{ title }</h1>
        <h3>{ subtitle }</h3>
    </Jumbotron >

export default Header
\end{minted}

\subsection{Footer.js}

\begin{minted}{javascript}
import React from 'react'

// Ohne Destructuring.
const Footer = props => 
    <section>
        { props.message }
    </section>

export default Footer
\end{minted}

\subsection{Message.js}

\begin{minted}{javascript}
import React from 'react'
import { Alert } from 'reactstrap'

/**
 * Diese Komponente zeigt eine Meldung an den Benutzer an. Falls die Nachricht 
 * null oder ein Leerstring ist, wird die Komponente nicht angezeigt.
 * 
 * @param {string} message Die Nachricht, welche angezeigt werden soll
 * @param {bool} isError  Falls True, eine Fehlermeldung, sonst eine Infomeldung (Default: Fehlermeldung)
 */
const Message = ({ message, isError = true }) => 
    message ? <Alert color={ isError ? 'danger' : 'info' }>{ message }</Alert> : null

export default Message
\end{minted}

\subsection{Loader.js}

\begin{minted}{javascript}
import React from 'react'
import loader from './images/loader.gif'

const style = { display: 'block', margin: 'auto' }

const Loader = () => 
    <img style={ style } src={loader} height="128" alt="Loading..."/>

export default Loader
\end{minted}


\subsection{Dialog.js}

\begin{minted}{javascript}
import React, { useState, useEffect } from 'react'
import _ from 'lodash'
import { Button, Modal, ModalHeader, ModalBody, Form, FormGroup, Col } from 'reactstrap'
import InputGroup from './InputGroup'

/**
 * Modale Dialog Komponente.
 * 
 * @param buttonLabel Label des Buttons in der Tabelle
 * @param title Der Titel des Dialogs
 * @param actionButtonLabel Label des 'Submit' Buttons
 * @param questionnaire Der Questionnaire
 * @param isReadOnly True, wenn die Input Felder im Dialog nicht editierbar sind, false sonst (Default: false)
 * @param actionFn Die Funktion, welche aufgerufen wird, wenn der Dialog geschlossen wird (Save, Close)
 * @param css Die CSS Klasse für den Button in der Tabelle
 */
const Dialog = ({ buttonLabel, title, actionButtonLabel, questionnaire: qx, isReadOnly = false, actionFn, css }) => {

    let [showModal, setShowModal] = useState(false)
    let [questionnaire, setQuestionnaire] = useState(qx)
    
    /**
     * Wir beobachten hier qx. qx ist der Questionnaire, der als props in die Komponente
     * hinein gegeben wird. Wenn sich qx ändert, wollen wir diese Änderung übernehmen.
     */
    useEffect(() => {
        setQuestionnaire(qx)
    }, [qx])

    const change = event => 
        setQuestionnaire({ ...questionnaire, [event.target.name]: event.target.value })

    const close = () => 
        setShowModal(false)

    const open = () => 
        setShowModal(true)

    /**
     * Diese Funktion wird aufgerufen, wenn Save, oder Close gedrückt wird.
     * Sie übernimmt die actionFn, die mittels props in diese Komponente 
     * übergeben wird. Falls es keine actionFn gibt, wird identity verwendet.
     * Diese Funktion schliesst nach dem Aufruf der actionFn den Dialog.
     * 
     * @param questionnaire Der Questionnaire
     * @param actionFn Die Action-Funktion (update, create)
     */
    const onAction = (questionnaire, actionFn) => {
        (actionFn || _.identity)(questionnaire)
        close()
    }
    
    return <div>
        <Button color={ css || 'secondary'} onClick={ open }
            className='float-right'>{ buttonLabel }</Button>
        <Modal isOpen={ showModal } toggle={ close } size='lg' autoFocus={false}>
            <ModalHeader toggle={ close } >
                { title }
            </ModalHeader>
            <ModalBody>
                <Form>
                    <InputGroup 
                        label='Title'
                        id='formTitle'
                        changeFn={ change }
                        name='title'
                        value={ questionnaire.title }
                        isReadOnly={ isReadOnly }
                    />

                    <InputGroup 
                        label='Description'
                        id='formDescription'
                        changeFn={ change }
                        name='description'
                        value={ questionnaire.description }
                        isReadOnly={ isReadOnly }
                    /> 

                    <FormGroup>
                        <Col className='clearfix' style={{ padding: '.2rem' }}>
                            <Button className='float-right' color='secondary'
                                onClick={ _.partial(onAction, questionnaire, actionFn) }>{ actionButtonLabel }</Button>
                        </Col>
                    </FormGroup>
                </Form>
            </ModalBody>
        </Modal>
    </div>
}

export default Dialog
\end{minted}

\subsection{InputGroup.js}

\begin{minted}{javascript}
import React from 'react'
import { FormGroup, Label, Col, Input } from 'reactstrap'

/**
 * Zeigt ein Label und ein Input Field an.
 * 
 * @param {string} label Das Label des Input Field
 * @param {string} id Die id, welche das Label und das Input Field verbindet (for - id)
 * @param {function} changeFn Die onChange Funktion des Input Fields
 * @param {string} name Der Name des Input Fields
 * @param {string} value Der Value des Input Fields
 * @param {bool} isReadOnly True, wenn das Input Field readonly ist, false sonst
 */
const InputGroup = ({ label, id, changeFn, name, value, isReadOnly }) => 
    <FormGroup row>
        <Label md={ 2 } for={ id }>
            { label }
        </Label>
        <Col md={ 10 }>
            <Input 
                type='text' 
                id={ id }
                onChange={ changeFn }
                name={ name }
                value={ value }
                plaintext={ isReadOnly }
                disabled={ isReadOnly } />
        </Col>
    </FormGroup>

export default InputGroup
\end{minted}


\subsection{QuestionnaireContainer.js}
\begin{minted}{javascript}
import React, { useEffect } from 'react'
import { useSelector, useDispatch } from 'react-redux'
import _ from 'lodash'
import QuestionnaireTable from './QuestionnaireTable'
import QuestionnaireCreateDialog from './QuestionnaireCreateDialog'
import doFetch from '../network/NetworkUtil'
import Message from '../app/Message'
import Loader from '../app/Loader'

const headers = { headers: { 'Content-Type': 'application/json; charset=utf-8' } }

/**
 * Die Questionnaire Funktionalität (Crerate, Tabelle der Questionnaires).
 * 
 * @param {string} serverUrl Die URL des Backends
 */
const QuestionnaireContainer = ({ serverUrl }) => {

    const qs = useSelector(state => state.qs, _.isEqual)
    const error = useSelector(state => state.error, _.isEqual)
    const message = useSelector(state => state.message, _.isEqual)
    const loading = useSelector(state => state.loading, _.isEqual)
    const dispatch = useDispatch()

    const readAll = () => {
        dispatch(
            doFetch({
                url: serverUrl,
                actionType: 'READ_QUESTIONNAIRES',
                errorText: 'Not Found'
            })
        )
    }

    useEffect(readAll, [])

    const create = async questionnaire => {
        dispatch(
            doFetch({
                url: serverUrl,
                requestObject: { method: 'POST', body: JSON.stringify(questionnaire), ...headers },
                actionType: 'CREATE_QUESTIONNAIRES',
                errorText: 'Creation failed.'
            })
        )
    }

    const update = questionnaire => {
        dispatch(
            doFetch({
                url: `${ serverUrl }/${ questionnaire.id }`,
                requestObject: { method: 'PUT', body: JSON.stringify(questionnaire), ...headers },
                actionType: 'UPDATE_QUESTIONNAIRES',
                errorText: 'Not found, or update failed.'
            })
        )
    }

    const _delete = id => {
        dispatch(
            doFetch({
                url: `${ serverUrl }/${ id }`,
                requestObject: { method: 'DELETE' },
                actionType: 'DELETE_QUESTIONNAIRES',
                errorText: 'Not found, or delete failed.'
            })
        )
        // Ist nötig, wenn wir die REST Schnittstelle nicht verändern wollen.
        // Besser wäre es, wenn wir das gelöschte Questionnaire zurückgeben
        // und dann im NetworkUtil die ID mittels Action an den Reducer mitgeben.
        readAll()
    }

    const renderMessage = () => 
        error ? <Message message={ message } /> : null

    const renderQuestionnaireTable = (qs, update, _delete) => 
        loading ? <Loader /> : <QuestionnaireTable qs={ qs } update={ update } _delete={ _delete } />

    return <div>
            <QuestionnaireCreateDialog create={ create } />
            <h3>Questionnaires</h3>
            { renderMessage() }
            { renderQuestionnaireTable(qs, update, _delete) }
        </div>
}

export default QuestionnaireContainer
\end{minted}

\subsection{QuestionnaireCreateDialog.js}
\begin{minted}{javascript}
import React from 'react'
import Dialog from './Dialog'

/**
 * Erzeugt einen Questionnaire.
 * 
 * @param {function} create Die create Function
 */
const QuestionnaireCreateDialog = ({ create }) => {
    return <Dialog 
        buttonLabel='Add Questionnaire' 
        title='Add Questionnaire' 
        actionButtonLabel='Save'
        questionnaire={ { title: '', description: '' } }
        actionFn={ create } 
        css='success' />
}

export default QuestionnaireCreateDialog
\end{minted}


\subsection{QuestionnaireShowDialog.js}
\begin{minted}{javascript}
import React from 'react'
import Dialog from './Dialog'

/**
 * Zeigt einen Questionnaire (readonly) an.
 * 
 * @param {object} questionnaire Der Questionnaire, der angezeigt wird
 */
const QuestionnaireShowDialog = ({ questionnaire }) => 
    <Dialog 
        buttonLabel='Show' 
        title='Show Questionnaire' 
        actionButtonLabel='Close' 
        questionnaire={ questionnaire }  
        isReadOnly={ true } />

export default QuestionnaireShowDialog
\end{minted}

\subsection{QuestionnaireTable.js}
\begin{minted}{javascript}
import React from 'react'
import _ from 'lodash'
import { Table } from 'reactstrap'
import QuestionnaireTableElement from './QuestionnaireTableElement'

/**
 * Die Tabelle der Questionnaires und die dazugehörigen 
 * Controls (Show, Edit, Delete).
 * 
 * @param {array} qs Die Liste der Questionnaires
 * @param {function} update Die Funktion zum Updaten
 * @param {function} _delete Die Funktion zum Löschen
 */
const QuestionnaireTable = ({ qs, update, _delete }) => 
    <Table hover>
        <tbody>
        { 
            _.map(qs, questionnaire => 
                <QuestionnaireTableElement 
                    key={ questionnaire.id } 
                    questionnaire={ questionnaire }
                    update={ update }
                    _delete={ _delete } />
            ) 
        }
        </tbody>
    </Table>

export default QuestionnaireTable
\end{minted}

\subsection{QuestionnaireTableElement.js}
\begin{minted}{javascript}
import React from 'react'
import _ from 'lodash'
import { Button } from 'reactstrap'
import QuestionnaireShowDialog from './QuestionnaireShowDialog'
import QuestionnaireUpdateDialog from './QuestionnaireUpdateDialog'

/**
 * Eine Zeile in der Tabelle.
 * 
 * @param {object} questionnaire Der Questionnaire, der angezeigt werden soll
 * @param {function} update Die Funktion zum Updaten
 * @param {function} _delete Die Funktion zum Löschen 
 */
const QuestionnaireTableElement = ({ questionnaire, update, _delete }) => (
    <tr key={ questionnaire.id } >
        <td>{ questionnaire.id }</td>
        <td>{ questionnaire.title }</td>
        <td>{ questionnaire.description }</td>
        <td><QuestionnaireShowDialog questionnaire={ questionnaire } /></td>
        <td><QuestionnaireUpdateDialog update={ update } questionnaire={ questionnaire } /></td>
        <td><Button color='danger' 
                    onClick={ _.partial(_delete, questionnaire.id) } 
                    className='float-right' >Delete</Button>
        </td>
    </tr>
)

export default QuestionnaireTableElement
\end{minted}

\subsection{QuestionnaireUpdateDialog.js}
\begin{minted}{javascript}
import React from 'react'
import Dialog from './Dialog'

/**
 * Passt einen bestehenden Questionnaire an.
 * 
 * @param {object} questionnaire Der Questionnaire, der angepasst werden soll
 * @param {function} update Die update Funktion
 */
const QuestionnaireUpdateDialog =  ({ questionnaire, update }) => 
    <Dialog 
        buttonLabel='Edit' 
        title='Edit Questionnaire' 
        actionButtonLabel='Save'
        questionnaire={ questionnaire } 
        actionFn={ update } 
        css='primary' />

export default QuestionnaireUpdateDialog

\end{minted}

\subsection{reducers.js}

\begin{minted}{javascript}
import _ from 'lodash'

const REDUCERS = {
    'READ_QUESTIONNAIRES': (state, action) => ({ ...state, qs: action.data }),
    'CREATE_QUESTIONNAIRES': (state, action) => ({ ...state, qs: [...state.qs, action.data ] }),
    'UPDATE_QUESTIONNAIRES': (state, action) => ({ ...state, qs: _.map(state.qs, q => q.id === action.data.id ? action.data : q) }),
    'DELETE_QUESTIONNAIRES': (state, action) => ({ ...state, qs: _.reject(state.qs, { id: action.data }) }),
    'LOADING': (state, action) => ({ ...state, loading: action.data }),
    'MESSAGE': (state, action) => ({ ...state, message: action.data }),
    'ERROR': (state, action) => ({ ...state, error: action.data }),
    'CONFIG': (state, action) => ({ ...state, config: action.data })
}

const reducer = (state, action) => _.get(REDUCERS, action.type, _.identity)(state, action)

export default reducer
\end{minted}


\subsection{NetworkUtil.js}
\begin{minted}{javascript}
/**
 * Eine Funktion, welche den Fetch-Request absetzt und die entsprechenden
 * Actions an den Store dispatched.
 * 
 * @param {string} url Der URL
 * @param {object} requestObject Das Objekt mit den Headern, Methoden (wenn nicht GET) (Optional)
 * @param {string} actionType Der Action-Type
 * @param {string} errorText Den Fehlertext
 * @return {function} Die Dispatch Funktion für Redux-Thunk
 */
const doFetch = ({ url, requestObject, actionType, errorText }) => 
    async dispatch => {
        try {
            dispatch({ type: 'LOADING', data: true })
            const response = await fetch(url, requestObject)
            if(response.ok) {
                // Mit leerem Response umgehen:
                const json = response.status !== 204 ? await response.json() : null
                dispatch({ type: actionType, data: json })
                dispatch({ type: 'ERROR', data: false })
                dispatch({ type: 'MESSAGE', data: '' })
            }
            else {
                throw new Error(`${ errorText }: ${ response.status }`)
            }
        }
        catch(error) {
            dispatch({ type: 'ERROR', data: true })
            dispatch({ type: 'MESSAGE', data: error.message })
        }
        dispatch({ type: 'LOADING', data: false })
    }

export default doFetch
\end{minted}